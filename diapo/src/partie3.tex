\section{Calcul des distances pour la chaîne ROP}

% --- Slide 1 : Gadget pop rdi ---
\begin{frame}[fragile]{Identification du gadget \texttt{pop rdi ; ret}}
    Pas d'instruction explicite \texttt{pop rdi} dans le binaire, mais :

    \begin{lstlisting}[language=bash]
0x555555555562 <+98>:    pop    %r15   ; opcode: 41 5f
0x555555555564 <+100>:   ret            ; opcode: c3
    \end{lstlisting}

    \begin{itemize}
        \item \texttt{pop r15} = \code{41 5f}, \texttt{pop rdi} = \code{5f}
        \item En sautant 1 octet : le CPU exécute \code{5f c3} = \texttt{pop rdi ; ret}
    \end{itemize}
    \[ \text{Adresse gadget} = \texttt{0x555555555562} + 1 = \textbf{\texttt{0x555555555563}} \]
\end{frame}

% --- Slide 2 : Tableau et calculs ---
\begin{frame}{Calcul des offsets}
    \begin{table}
        \centering
        \begin{tabular}{|l|c|}
            \hline
            \textbf{Élément} & \textbf{Adresse} \\
            \hline
            Référence -- Saved RIP & \texttt{0x...54f3} \\
            \hline
            Gadget -- \texttt{pop rdi ; ret} & \texttt{0x...5563} \\
            \hline
            \texttt{system@plt} & \texttt{0x...5100} \\
            \hline
        \end{tabular}
    \end{table}

    \vspace{0.3cm}
    \textbf{Distances relatives :}
    \begin{itemize}
        \item $\Delta_{\text{Gadget}} = \texttt{0x5563} - \texttt{0x54f3} = \textbf{+0x70}$ (112 octets)
            \begin{itemize}
                \item[$\rightarrow$] $< 256$ : atteignable en modifiant 1 seul octet (LSB)
            \end{itemize}
        \item $\Delta_{\text{System}} = \texttt{0x5100} - \texttt{0x54f3} = \textbf{-0x3F3}$ (--1011 octets)
            \begin{itemize}
                \item[$\rightarrow$] Offset constant, calculé à partir du leak d'adresse
            \end{itemize}
    \end{itemize}
\end{frame}
