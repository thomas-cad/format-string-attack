\section{Analyse préliminaire du programme}

% --- Slide 1 : Test de détection ---
\begin{frame}[fragile]{Détection de la vulnérabilité}
    \textbf{Payload injecté :} \code{AAAA \%p \%p \%p \%p \%p}

    \begin{lstlisting}[language=bash]
$ ./vuln
Please Insert an IP address to ping:
AAAA %p %p %p %p %p
AAAA 0x5acad81676b1 0xfbad2288 0x7e170491ba91 0x5acad81676c4 0x410
    \end{lstlisting}

    \begin{itemize}
        \item Le programme \textbf{interprète} les spécificateurs au lieu de les afficher
        \item Les valeurs hexadécimales affichées sont des \textbf{données de la pile}
        \item[$\Rightarrow$] Vulnérabilité \textbf{Format String} confirmée
    \end{itemize}
\end{frame}

% --- Slide 2 : Cause et capacités ---
\begin{frame}[fragile]{Cause et capacités d'exploitation}
    \textbf{Origine :} l'entrée utilisateur est passée directement comme format string

    \begin{columns}[T]
        \begin{column}{0.48\textwidth}
            \textbf{Vulnérable :}
            \begin{lstlisting}[language=C]
printf(user_input);
            \end{lstlisting}
        \end{column}
        \begin{column}{0.48\textwidth}
            \textbf{Sécurisé :}
            \begin{lstlisting}[language=C]
printf("%s", user_input);
            \end{lstlisting}
        \end{column}
    \end{columns}

    \vspace{0.4cm}
    \textbf{Deux vecteurs d'attaque :}
    \begin{itemize}
        \item \textbf{Lecture} (\code{\%p}, \code{\%x}, \code{\%s}) : fuite d'adresses mémoire $\rightarrow$ contournement ASLR/PIE
        \item \textbf{Écriture} (\code{\%n}, \code{\%hn}, \code{\%hhn}) : écriture en mémoire $\rightarrow$ détournement du flux d'exécution
    \end{itemize}
\end{frame}
