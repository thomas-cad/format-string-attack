\section{Construction de la boucle d'exploitation}

% --- Slide 1 : Mécanisme LSB Overwrite ---
\begin{frame}[fragile]{Mécanisme de réinvocation (LSB Overwrite)}
    \textbf{Principe :} modifier le dernier octet du Saved RIP pour reboucler sur \texttt{makePing}

    \begin{lstlisting}[language=bash]
0x5555555554ee <+24>:  call  makePing   ; <-- cible (LSB = 0xee)
0x5555555554f3 <+29>:  mov   $0x0,%eax  ; <-- Saved RIP actuel (LSB = 0xf3)
    \end{lstlisting}

    \vspace{0.2cm}
    \textbf{Pourquoi seulement le LSB ?}
    \begin{itemize}
        \item PIE/ASLR randomise l'adresse de base, mais les 12 bits de poids faible restent \textbf{constants} (alignement 4 Ko)
        \item \code{\%hhn} écrase uniquement 1 octet $\rightarrow$ pas besoin de connaître l'adresse complète
        \item Remplacement : \texttt{0xf3} $\rightarrow$ \texttt{0xee} (distance = 5 octets)
    \end{itemize}
\end{frame}

% --- Slide 2 : Algorithme ---
\begin{frame}{Algorithme de la boucle d'attaque}
    \textbf{3 itérations successives :}

    \vspace{0.3cm}
    \begin{enumerate}
        \item \textbf{Itération 1} -- Préparation de \texttt{system}
            \begin{itemize}
                \item Écrire adresse de \texttt{system} en \textbf{RIP+16} (mot 29)
                \item Écraser LSB du Saved RIP avec \texttt{0xee} $\rightarrow$ \texttt{makePing} redémarre
            \end{itemize}

        \vspace{0.2cm}
        \item \textbf{Itération 2} -- Préparation de \texttt{"/bin/sh"}
            \begin{itemize}
                \item Écrire adresse de \texttt{"/bin/sh"} en \textbf{RIP+8} (mot 28)
                \item Écraser LSB du Saved RIP avec \texttt{0xee} $\rightarrow$ \texttt{makePing} redémarre
            \end{itemize}

        \vspace{0.2cm}
        \item \textbf{Itération 3} -- Déclenchement
            \begin{itemize}
                \item Écraser Saved RIP avec l'adresse du gadget (LSB = \texttt{0x63})
                \item[$\Rightarrow$] \texttt{pop rdi ; ret} $\rightarrow$ \texttt{system("/bin/sh")} $\rightarrow$ \textbf{shell obtenu}
            \end{itemize}
    \end{enumerate}
\end{frame}
