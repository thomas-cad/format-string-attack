\section{Extraction d’informations à l’aide d’un débogueur}

L'exploitation d'une vulnérabilité binaire, et plus particulièrement la construction d'une chaîne ROP, nécessite une cartographie précise de la mémoire. L'utilisation du débogueur \texttt{GDB} nous permet d'analyser l'état de la pile et des registres en temps réel.

\subsection{Session de débogage et collecte de données}

Pour déterminer l'emplacement exact de notre buffer et de l'adresse de retour, nous avons exécuté le programme vulnérable sous \texttt{GDB} en injectant un motif reconnaissable (\texttt{AAAA}) suivi de plusieurs offsets stratégiques (7, 16, 26 et 27).

\textbf{Commande GDB utilisée :} \code{run} puis injection de : \\
\code{AAAA | Off7=\%7\$p | Off16=\%16\$p | Off26=\%26\$p | Off27=\%27\$p}

\begin{lstlisting}[caption={Sortie GDB montrant l'état de la pile}, language=bash]
(gdb) run
Starting program: /home/thomas/work/devoir2026/vuln 
...
Checking ping is on the system...
/usr/bin/ping
Please Insert an IP address to ping: 
AAAA | Off7=%7$p | Off16=%16$p | Off26=%26$p | Off27=%27$p
AAAA | Off7=0x7fffffffdcc8 | Off16=0x4f207c2041414141 | Off26=0x7fffffffdce0 | Off27=0x5555555554f3
[Detaching after vfork from child process ...]
\end{lstlisting}

\subsection{Analyse des données exfiltrées}

L'architecture x86\_64 place les premiers arguments dans les registres. Les suivants sont sur la pile. L'analyse des valeurs retournées nous permet de situer notre entrée utilisateur.

\subsubsection{Analyse du Mot n°16 : \texttt{0x...41414141}}
La valeur obtenue à l'offset 16 est \texttt{0x4f207c2041414141}.
\begin{itemize}
    \item \textbf{Observation :} Les 4 octets de poids faible (Little Endian) sont \texttt{0x41414141}.
    \item \textbf{Interprétation :} Cela correspond au code ASCII de "AAAA". 
    \item \textbf{Conclusion :} Contrairement à l'offset 7 (qui est simplement le sommet de la pile \texttt{RSP}), notre buffer d'entrée commence réellement à l'\textbf{offset 16}. C'est notre point de référence (Offset 0 de notre payload).
\end{itemize}

\subsubsection{Analyse du Mot n°27 : \texttt{0x5555555554f3}}
La valeur obtenue est \texttt{0x5555555554f3}.
\begin{itemize}
    \item \textbf{Observation :} Il s'agit d'une adresse du segment de code (\texttt{.text}), typique d'un exécutable position-indépendant (PIE activé).
    \item \textbf{Identification (Saved RIP) :} Cette adresse correspond à l'\textbf{adresse de retour} vers la fonction \texttt{main}.
    \item \textbf{Preuve par le calcul :} Le désassemblage montre que l'appel à \texttt{makePing} se termine à l'adresse \texttt{...54ee}.
    \[ 0x54f3 - 0x5 = 0x54ee \]
    L'adresse obtenue dans GDB correspond exactement à l'instruction suivant l'appel. C'est la valeur que nous devrons écraser pour détourner le flux d'exécution.
\end{itemize}

\subsection{Organisation de la pile (Stack Mapping)}

Grâce à ces nouvelles données, nous pouvons reconstituer la cartographie exacte de la pile :

\begin{table}[H]
    \centering
    \begin{tabular}{|c|c|l|l|}
        \hline
        \textbf{Offset printf} & \textbf{Adresse (Exemple)} & \textbf{Contenu} & \textbf{Description} \\
        \hline
        \%7\$p & \texttt{0x7fffffffdcc8} & Stack Pointer & Sommet de la pile (RSP) \\
        \hline
        ... & ... & ... & Padding et variables locales \\
        \hline
        \textbf{\%16\$p} & \texttt{0x41414141} & \textbf{Buffer Start} & \textbf{Début de notre payload} \\
        \hline
        ... & ... & ... & Données utilisateur \\
        \hline
        \textbf{\%26\$p} & \texttt{0x7fffffffdce0} & \textbf{Saved RBP} & Pointeur de base sauvegardé \\
        \hline
        \textbf{\%27\$p} & \texttt{0x5555555554f3} & \textbf{Saved RIP} & Adresse de retour vers main \\
        \hline
    \end{tabular}
    \caption{Organisation de la pile validée par GDB}
    \label{tab:stack_mapping}
\end{table}

\paragraph{Calcul des distances critiques :}
Pour atteindre nos cibles depuis le début de notre buffer (Offset 16), nous devons traverser :

\begin{itemize}
    \item \textbf{Vers Saved RBP (Offset 26) :} $26 - 16 = 10$ mots.
    \[ 10 \times 8 \mathrm{ octets} = \mathbf{80 \mathrm{ octets}} \]
    (Correspond à la variable \texttt{dist\_buffer\_rbp} du script).
    
    \item \textbf{Vers Saved RIP (Offset 27) :} $27 - 16 = 11$ mots.
    \[ 11 \times 8 \mathrm{ octets} = \mathbf{88 \mathrm{ octets}} \]
\end{itemize}